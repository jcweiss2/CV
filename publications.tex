\begin{bibsection}

\item  Wingrove et al.  Using machine learning to predict primary care and advance workforce research. \emph{Annals of Family Medicine}, 2020. 

\item  \textbf{Weiss JC}. Machine learning for clinical risk: wavelet reconstruction networks for marked point processes. Under review, 2019.
  
\item  Fillmore N, Goryachev S, and \textbf{Weiss JC}. Hypersphere clustering to characterize healthcare providers using prescriptions and procedures from Medicare claims data. Proceeding of the \emph{American Medical Informatics Association (AMIA) Annual Symposium}, 2019.
  
\item  Lo Ciganic et al. Evaluation of machine-learning algorithms for predicting opioid overdose risk among Medicare beneficiaries with opioid prescriptions. \emph{JAMA Network Open}, 2019.
  
\item  Lo Ciganic et al. Using machine learning to predict risk of opioid overdose in Medicare. \emph{Pharmacology and Drug Safety}, 2019. 

\item  Seymour, Christopher W., et al. Derivation, validation, and potential treatment implications of novel clinical phenotypes for sepsis. \emph{Journal of the American Medical Association (JAMA)}, 2019.

\item  Kleiman R, Kuusisto F, Ross I, Peissig PL, Stewart R, Page CD, and \textbf{Weiss JC}. Machine learning assisted discovery of novel predictive lab tests sing electronic health record data. Proceedings of the \emph{American Medical Informatics Association (AMIA) Informatics Summit}. San Francisco, 2019.

\item  Lo-Ciganic W, Huang JL, Zhang HH, \textbf{Weiss JC}, Wu Y, Kwoh CK, Donohue JM, Cochran G, Gordon AJ, Malone DC, Kuza CC, Gellad WF. Use of Machine Learning to Predict Risk of Opioid Overdose among Medicare Beneficiaries Prescribed Opioids. \emph{JAMA Open}, 2018.
  
\item  \textbf{Weiss JC}. Piecewise-constant parametric distribution approximations for survival learning. Machine Learning for Healthcare Conference, \emph{Proceedings of Machine Learning Research (PMLR)}, 2017.

\item  \textbf{Weiss JC}, Kuusisto F, Boyd K, Liu J, and Page CD. Machine learning for treatment assignment: improving individualized risk attribution. \textit{American Medical Informatics Association (AMIA) Annual Symposium}. San Francisco, 2015.

\item  Lantz E, \textbf{Weiss JC}, Page CD, Schmelzer J, Berg R, Yale S, Miller A, and Burmester J. Using electronic health records to predict therapeutic warfarin dose. \textit{American Medical Informatics Association Joint Summit on Translational Science}, 2015.

\item  \textbf{Weiss JC}, Natarajan S, and Page D. Learning to reject sequential importance steps for continuous-time Bayesian networks. \textit{Association for the Advancement of Artificial Intelligence (AAAI)}. Austin, 2015.

\item  \textbf{Weiss JC}. Statistical timeline analysis for electronic health records. University of Wisconsin-Madison, 2014. PhD Thesis.

\item  \textbf{Weiss JC} and Page D. Forest-based point processes for event prediction from electronic health records. \textit{European Conference on Machine Learning (ECML-PKDD)}, Prague, CZ, 2013.

\item  \textbf{Weiss JC}, Natarajan S, Page D. Multiplicative forests for continuous-time processes. \textit{Neural Information Processing Systems (NeurIPS)}, Lake Tahoe, 2012.

\item  \textbf{Weiss JC}, Natarajan S, Peissig P, McCarty C, and Page D. Machine learning for personalized medicine: predicting primary myocardial infarction from electronic health records. \textit{AI Magazine}, Winter 2012.

\item  \textbf{Weiss JC}, Natarajan S, Peissig P, McCarty C, and Page D. Statistical relational learning to predict primary myocardial infarction from electronic health records. Innovative Applications of Artificial Intelligence (IAAI). Toronto, 2012.

\item  Lovasi GS, \textbf{Weiss JC}, Hoskins R, Whitsel EA, Rice K, Erickson CF, and Psaty BM. Comparing a single-stage geocoding method to a multi-stage geocoding method: how much and where do they disagree. \textit{International Journal of Health Geographics}.\textit{16};6:12, 2007.
  
\end{bibsection}

